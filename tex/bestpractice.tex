``Best practice'' är ett begrepp som alla har en bild utav vad det innebär inom ens eget område.
Tyvärr är det i stort sett aldrig är så att det finns en enhetlig bild inom varje område, sällan ens
på varje arbetsplats, eller ens mellan team på samma arbetsplats.

När det gäller systemadministration så är det givetvis likadant, och det viktigaste är i sedvanlig ordning dokumentation.
Har man en god dokumentation så kan resten lösas med resurser. Ansible är inte avsett att vara ett dokumentationsverktyg,
men det kan utgöra ett utmärkt komplement till exempelvis en wiki.

Men, det här är inte en guide till systemadministration eller dokumentation, det är en guide till Ansible, så låt 
oss fokusera på detta.

Best practice är till viss del beroende på hur vad man använder Ansible till och hur stora och varierade miljöer
man har. Den här guiden försöker främst hålla ett scenario tanken där vi har en varierad miljö med olika 
linuxdistributioner och där vi samtidigt söker en skalbarhet. Att hantera en eller ett par maskiner klarar man utan
att blanda in exempelvis Ansible (även om det är praktiskt att kunna rulla om maskiner smidigt), men att hålla
tiotals, hundratals, eller till och med tusentals klienter i ordning på ett enhetligt sätt kräver verktyg, och varje
miljö börjar med en första maskin.

Att använda någon form av versionshantering är att se som lite av ett måste när samarbete mellan administratörer
förekommer, och även en automatiseringskedja när miljön växer. Det är dock två faktorer vi inte kommer att beröra
här då det är utanför denna guides ``scope'', men möjligheterna finns.

\section{Filstruktur}
En tydlig filstruktur hjälper skalbarhet, läsbarhet och även implementation av versionshantering (ex. Git). Det 
finns flera olika modeller som är bra, men här lyfts bara en av den enkla anledningen att det är en modell 
undertecknad själv har använt framgångsrikt och för att omfattningen av denna guide skall hållas begränsad. Den 
är en av de modeller som rekommenderas av RedHat, så det är inget hemkok på något sätt. Vi börjar med att illustrera
den:

\begin{verbatim}
./
  hosts                   # global inventory file
  
  group_vars/
    <groupname>           # variables for groups
  host_vars/
    <hostname>            # variables for specific systems
  
  library/                # custom modules
  filter_plugins/         # custom filter plugins
  
  playbooks/              # directory for playbooks
    playbook1.yml

  roles/                  
    common/               # this hierarchy represents a "role"
        tasks/            #
            main.yml      #  <-- tasks file can include smaller files if warranted
        handlers/         #
            main.yml      #  <-- handlers file
        templates/        #  <-- files for use with the template resource
            ntp.conf.j2   #  <------- templates end in .j2
        files/            #
            bar.txt       #  <-- files for use with the copy resource
            foo.sh        #  <-- script files for use with the script resource
        vars/             #
            main.yml      #  <-- variables associated with this role
        defaults/         #
            main.yml      #  <-- default lower priority variables for this role
        meta/             #
            main.yml      #  <-- role dependencies

\end{verbatim}

Inventory-filer har vi redan berört, och generellt bör endast en sådan finnas och hållas så läsbar som möjligt.
Det är även i inventoryt som grupperingar av maskinfunktioner bör ske på ett vettigt sätt.

\texttt{library} samt \texttt{filter\_plugins} är för egenskrivna moduler och plugins. 
Det är inget vi kommer att beröra här dock.

\texttt{group\_vars} och \texttt{host\_vars} innehåller variabelfiler för nodgrupper och enskilda noder.
Här bör vi tänka på att vi vill lägga så mycket specifik konfiguration av enskilda maskiner/grupper här. 

\texttt{roles} är vad vi skulle kunna likna vid programmeringsfunktioner och strukturer. Roller bör vara så
generellt skrivna som möjligt, utan nod-specifika variabler, idempotenta, begränsade i dess funktion och 
om möjligt oberoende av distribution (eller med stöd för alla berörda distributioner).

\texttt{playbooks} är den mapp där vi generellt lagrar de playbooks vi skriver. En playbook kan vara hur begränsad
eller omfattande man önskar, men generellt bör de vara begränsade antingen i funktion eller gruppering av noder.
Kan man inte på mindre än en minut få en god överblick över vad en playbook gör bör man fundera på hur man kan
bryta ner den i mindre.
