Den stora bredd på välskrivna moduler som skeppas med Ansible är vad som separerar Ansible från alternativen.
Den här guiden är på intet sätt avsedd att omfatta ens en bråkdel av modulerna, men vi kommer att nämna de
vanligaste och mest grundläggande för att "komma igång". Inte heller tas allting upp som kan göras med respektive
modul. Därav kan du härmed se dig permanent refererad till Ansibles dokumentation. Även jag som jobbar med
Ansible har oftast en eller flera tabbar med dokumentation för olika moduler uppe i min webbläsare. 
Dokumentationen finns där av en anledning -- att användas. Att memorera annat än namnen på de allra vanligaste är
ett slöseri med tid i de flesta fall.

Varje modulsektion nedan kommer med ett mer eller mindre grundläggande exempel. I de fall det är mer än minimala
exempel så beror det på att jag antingen vill belysa en kraftfull användning av modulen eller att jag vill introducera något koncept.

\section{package}
Packagemodulen används för att installera paket med pakethanterare. De flesta pakethanterare har en egen modul som
är betydligt kraftfullare än \texttt{package}, men i nio av tio fall räcker denna modul mer än väl. Fördelen är att
vi inte behöver fundera på om systemet använder dpkg, yum, pacman eller whatever, vi behöver bara bry oss om vad 
paketet vi vill hantera heter. Ofta har paket samma namn mellan distar, men det kan skilja sig åt och då behöver
man täcka upp för det (se många av mina ansible-role-repon för exempel på detta). Betänk att pakethanterare ofta
behöver root-privilegier.

Modulen har och behöver två parametrar:
\texttt{name} ger namnet på det paket som ska hanteras.

\texttt{state} bestämmer om paketet i fråga ska vara installerat (\texttt{present}) eller inte (\texttt{absent}). De
flesta underliggande moduler stödjer även (\texttt{latest}).

\begin{verbatim}
- name: Install SELINUX libs if needed
  when: ansible_selinux and ansible_selinux.status == 'enabled'
  become: True
  package:
    name: "{{ item }}"
  with_items:
    - libselinux-python
    - policycoreutils-python
\end{verbatim}


\section{template}
Vi kommer att titta närmare på hur vi bygger templates senare, men till en början ska vi se hur vi använder modulen.


\section{service}

\section{shell}

\section{command}

\section{file}

\section{lineinfile}

\section{stat}

\section{group}

\section{user}
