Den stora bredd på välskrivna moduler som skeppas med Ansible är vad som separerar Ansible från alternativen.
Den här guiden är på intet sätt avsedd att omfatta ens en bråkdel av modulerna, men vi kommer att nämna de
vanligaste och mest grundläggande för att "komma igång". Inte heller tas allting upp som kan göras med respektive
modul. Därav kan du härmed se dig permanent refererad till Ansibles dokumentation. Även jag som jobbar med
Ansible har oftast en eller flera tabbar med dokumentation för olika moduler uppe i min webbläsare. 
Dokumentationen finns där av en anledning -- att användas. Att memorera annat än namnen på de allra vanligaste är
ett slöseri med tid i de flesta fall.

Varje modulsektion nedan kommer med ett mer eller mindre grundläggande exempel. I de fall det är mer än minimala
exempel så beror det på att jag antingen vill belysa en kraftfull användning av modulen eller att jag vill introducera något koncept.

\section{package}
Packagemodulen används för att installera paket med pakethanterare. De flesta pakethanterare har en egen modul som
är betydligt kraftfullare än \texttt{package}, men i nio av tio fall räcker denna modul mer än väl. Fördelen är att
vi inte behöver fundera på om systemet använder dpkg, yum, pacman eller whatever, vi behöver bara bry oss om vad 
paketet vi vill hantera heter. Ofta har paket samma namn mellan distar, men det kan skilja sig åt och då behöver
man täcka upp för det (se många av mina ansible-role-repon för exempel på detta). Betänk att pakethanterare ofta
behöver root-privilegier.

Modulen har och behöver två parametrar:
\texttt{name} ger namnet på det paket som ska hanteras.

\texttt{state} bestämmer om paketet i fråga ska vara installerat (\texttt{present}) eller inte (\texttt{absent}). De
flesta underliggande moduler stödjer även (\texttt{latest}).

\begin{verbatim}
- name: Install SELINUX libs if needed
  when: ansible_selinux and ansible_selinux.status == 'enabled'
  become: True
  package:
    name: "{{ item }}"
  with_items:
    - libselinux-python
    - policycoreutils-python
\end{verbatim}


\section{template}
Vi kommer att titta närmare på hur vi bygger templates senare, men till en början ska vi se hur vi använder modulen.

\begin{verbatim}
- name: configure ntp (ntpd)
  when: ( ansible_os_family == 'RedHat' and ansible_distribution_major_version == '6') or
        systemd_ntpd.stat.islnk is defined
  notify: restart ntp
  become: true
  template:
    src: "ntp.conf_{{ ansible_os_family }}{{ ansible_distribution_major_version }}.j2"
    dest: /etc/ntp.conf
  tags: NTP
\end{verbatim}

Det här är en task som bjuder på lite variabelanvändning (bara för demonstration av variabler samt konditionell 
användning, så \texttt{when}-delen är inte relevant för modulen i sig). Men, om vi ser till de modulrelevanta 
raderna så ser vi följande:

\begin{verbatim}
- name: configure ntp (ntpd)
  template:
    src: "ntp.conf.j2"
    dest: /etc/ntp.conf
\end{verbatim}

Vi har en template, \texttt{templates/ntp.conf.j2} på styrsystemet som vi använder för att konfigurera \texttt{/etc/ntp.conf} på noderna. Exemplet ovan är minimalt, men det finns fler användbara parametrar:

\texttt{backup: yes} skapar en backupfil med timestamp så att du kan återställa originalet om det skulle behövas.

\texttt{validate} erbjuder möjligheten att validera den nya filen med valt kommando innan vi skriver över den gamla.
Sökvägen skickas som \verb+%s+ (se exempel nedan).

Det kan även vara bra att känna till några andra parametrar, nämligen de som styr filrättigheter:

\texttt{owner} anger vilken user som ska ha rättigheter (per default är det den användare som ansible använder, d.v.s. den man ansluter som eller den man eleverar sig till (\texttt{root} i fallet \texttt{become: yes})).

\texttt{group} är som \texttt{owner} men för group.

\texttt{mode} agerar chmod, men har lite egenheter. Den accepterar symbolic mode (ex. \texttt{u+rwx}) och även
oktetter. Använder vi oss av en oktala värden behöver modulen dock veta det genom att vi inleder med 0, alternativt
sätter det inom enkla citattecken, ex. \texttt{0644, 01777} eller \texttt{'644'}. Vi kan även skicka \texttt{preserve} för att bibehålla befintliga rättigheter (endast Ansible 2.6 eller senare.)

Exempel:
\begin{verbatim}
- name: "Update sshd config safely to avoid locking yourself out."
  template:
    src: etc/ssh/sshd_config.j2
    dest: /etc/ssh/sshd_config
    owner: root
    group: root
    mode: '0600'
    validate: /usr/sbin/sshd -t -f %s
    backup: yes
\end{verbatim}

\section{service}
Den här modulen har vi sprungit på tidigare, och den stödjer BSD init, OpenRC, SysV, Solaris SMF, systemd och upstart. De viktigaste parametrarna är:

\texttt{name} - obligatorisk.

\texttt{enabled} - huruvida servicen skall autostarta. Boolean \texttt{yes/no}. Obligatorisk om inte \texttt{state} anges.

\texttt{state} - obligatorisk om inte \texttt{enabled} anges. De båda är inte exklusiva och påverkar inte varandra,
men endera eller båda krävs. Tar argumenten \verb+reloaded/restarted/running/started/stopped+, där de sista två är idempotenta och körs inte om de inte behövs. Märk väl att \verb+reloaded+ kommer att starta servicen om den inte 
är startad redan, oavsett vad init-systemet i fråga gör i vanliga fall.

\section{shell/command}
De här modulerna är i stort sett identiska, men \texttt{shell} kör kommandot i ett eget skall (\texttt{/bin/sh}) på 
aktuell nod. Generellt sett bör \texttt{command} användas.

Modulerna tar in kommandoraden som skall köras direkt efter modulcallen, och oftast vill vi ta ut ett resultat från 
körningen, vilket vi gör genom att tilldela tasken en registrerad variabel:

\begin{verbatim}
- name: return motd to registered variable
  command: cat /etc/motd
  register: mymotd
\end{verbatim}

Parametrar vi vanligen vill använda är \texttt{creates/removes} samt \texttt{chdir}. Sistnämnda ändrar directory alldeles innan kommandot körs, och de förstnämnda kollar huruvida given path existerar, och om den gör/inte gör det så körs kommandot. Att sätta parametern tar inte bort eller skapar filerna i sig, utan utgår från att kommandot skapar dessa.

\section{file}
Den här modulen kan skapa filer, kataloger, länkar, modifiera alla dessas rättigheter.
Den enda parameter som är obligatorisk är \texttt{path}, vilket givetvis avser sökvägen på aktuell nod.

Parametrarna \verb?owner, group, mode? fungerar som i \texttt{template}, och \texttt{recurse} applicerar rekursivt när en katalog är utpekad.

\texttt{src} kan användas för att skapa länkar (endast med \texttt{state=link} och \texttt{state=hard}) och tar då sökvägen till den fil vi vill länka till.

\texttt{state} är oftast med, även om det inte är ett krav (exempelvis kanske vi är säkra på att filen finns, men vi
vill endast modifiera ägarskapet). Det skadar dock oftast inte att ha den med även om vi vet att filen finns.
\texttt{state} tar ett av sex argument vilka bör vara självförklarande om vi vet hur länkar fungerar, kommandot 
\texttt{touch} fungerar och att en katalog är en fil (men en särskild sorts fil). Argumenten är då:
\verb=absent,directory,file,hard,link,touch= där \texttt{file} är implicit. Värt att notera är att \texttt{directory} kommer att skapa alla kataloger i pathen (jmf. \texttt{mkdir -p}).

\section{lineinfile}
Det här är en väldigt kraftfull men svårhanterad modul, tillsammans med dess systermodul \texttt{blockinfile}. Jag tänker inte försöka sammanfatta hur den används, men den förutsätter att användaren är i någon utsträckning bekant
med regex, och kan användas för att sätta eller ta bort rader i en textfil. Läs på i dokumentationen ordentligt och testa på någon testfil innan du kör den live.

\section{ini_file}

\section{stat}

\section{group}

\section{user}
