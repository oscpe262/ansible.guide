Den här guiden skrapar, återigen, bara på ytan av Ansibles fulla potential. Den bör dock ge en smidig väg in i
Ansibles värld, och öppna upp ögonen på läsaren för en helt ny värld (åtminstone om vederbörande aldrig använt
något dylikt verktyg tidigare). Förhoppningsvis kommer läsaren att fortsättningsvis fundera på "hur kan jag lösa
det här med Ansible" snarare än att göra enskilda förändringar på enskilda maskiner om och om igen bara för att 
"det är ju minst lika smidigt att copy-n-paste:a eller tmux:a".

Lycka till och glöm inte en sysadmins ledord: Var lat, och gör det du gör väl så behöver du aldrig göra det igen.
