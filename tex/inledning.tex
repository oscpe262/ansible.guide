\section{Introduktion}
\subsection{Vad är det här för dokument?}
Risken är överhängande att du blivit påtvingad detta dokument av en studiekamrat eller kollega av någon anledning. 
Troligtvis på grund av att denne insett hur smidigt Ansible är för konfiguration av klienter, 
alternativt har du hittat det själv och eventuellt undrar du varför du ska fortsätta läsa. Oavsett orsaken så bör vi reda upp några saker:

Den här guiden är inte avsedd att vara en komplett dokumentation för Ansible -- det vore redundant, vilket förvisso är bra, men det skulle innebära att jag skulle behöva hålla den uppdaterad frekvent.
Inte heller kommer Ansible att hjälpa dig tidsmässigt att konfigurera ett enskilt system. Ju färre system och ju mer sällan de konfigureras, desto mindre nytta har du av Ansible.
Men, Ansible är ett utmärkt verktyg för att komma in i ``sysadmintänket'' (gör inte två gånger det du kan scripta) utan att behöva lära sig shellscript eller Python. Givetvis är det bra att kunna dessa också, men det är tidsödande
om man huvudsakligen ska konfigurera vardagliga saker.
Den här guiden är främst utvecklad med avseende på kursen TDDI41 vid LiU, och kommer även att använda enskilda delar av kursen som exempel, även om jag kommer att hålla mig ifrån att använda fullständiga exempel.
Trots allt så tenderar man att lära sig mest av att sitta och laborera med det själv.

Om det är något du vill göra men som inte täcks här eller i boken oven, googla eller vänd dig till något lämpligt forum på webben. 
Det den här guiden syftar till är att snabbt få upp nya användare till en nivå där de kan börja använda Ansible för att lösa labbuppgifterna.

Om du springer på något som du känner borde ha tagits upp och som faller inom dokumentets syfte, kontakta mig gärna på guidens gitrepo \href{http://github.com/oscpe262/ansible.guide}{\url{http://github.com/oscpe262/ansible.guide}} och berätta.

\subsubsection{Hur ska jag läsa det här dokumentet?}
Det står läsaren givetvis fritt att läsa det precis som denne själv önskar. Ett varningens finger bör dock lyftas till den som ämnar hoppa runt i texten då vissa avsnitt bygger på tidigare avsnitt
.%, således rekommenderas att åtminstone läsa kapitel \ref{sec:intro}, \ref{sec:grunder} och \ref{sec:dokstruk} i sin helhet, framför allt för den som är helt obekant med \LaTeX.

\subsubsection{Kritik av TDDI41}
Jag läste kursen hösten 2016, ett år innan jag började jobba som systemadministratör.
Även om kursen det året hade problem framför allt med prestanda på labbsystemet så hade jag inte jättemycket att invända mot innehållet, inte minst eftersom jag inte visste vad jag kunde förvänta mig.
Rent innehållsmässigt bjöds det på många bra delar, men även en hel del utdaterade, vilket var bra när man kom ut i verkligheten med en massa förlegade system. Sällan kunde jag ana att det första jag 
skulle få användning utav var NIS-erfarenhet ...

Nåja, vad som var bristande med kursen var dock vad jag kallar ``sysadmintänk''. Det första som rekommenderades var att göra alla labbar manuellt först, den sista obligatoriska labben var att scripta allt och skriva tester,
och sedan göra om allt med exempelvis Puppet, Ansible eller dylikt för ett högre betyg. Dumt tänkt!

Skulle jag ha lagt upp det så skulle jag ha gjort tvärt om -- använd ett sådant verktyg först, för det är vad du vill använda i verkliga livet om du sysslar med just 'systemadministration'.
Ur det perspektivet är Ansible ett utmärkt verktyg, just för att det inte gör jobbet åt dig -- du måste förstå vad som ska göras -- men det besparar dig en massa scriptande.

\subsection{Vad är Ansible?}
Ansible är en öppen mjukvara för automatisering av bland annat mjukvaruprovisionering och konfigurationsstyrning. Ansible ansluter via SSH (eller PowerShell, eller andra API:er, men jag lägger här fokus vid styrning från och till linuxmaskiner) till en eller flera noder -- seriellt eller parallellt.

Det som skiljer Ansible från exempelvis Chef eller Puppet är att Ansible är agentlös. Alla ändringar pushas ut till noderna. Det går att göra hooks, inte minst via Ansible Tower (licensmjukvara), men det är inget som kommer att beröras här.

\subsection{Vad tjänar jag på att använda Ansible?}
Tid och läsbarhet. Tack vare att Ansible skeppas med transparenta moduler så är det lätt att följa YAML-koden över vad varje playbook eller role gör. 
Likaså är det lätt att skriva dessa och man behöver inte sitta och sköta loopning över noder, loopning över listor på paket, och så vidare. Likaså behöver vi inte fundera på idempotens i större utsträckning om vi gör rätt från början.
Det tar givetvis tid att lära sig Ansible, men skalar vi upp saker och ting så kommer det att löna sig.

\subsection{Övrigt}
TDDI41 använder sig av debianbaserade OS, vilket är något jag har använt i väldigt låg utsträckning utöver just i den kursen. Jag strävar efter att göra den här guiden så generell som möjligt, och i de fall det inte går
kommer jag att vända mig till min dokumentation från TDDI41. Det finns dock en överhängande risk att det kan finnas färgningar av RedHat i det hela, då jag främst jobbar med RHEL och CentOS, samt Fedora privat och på min jobbdator.
