I det här avsnittet ska vi titta på de vanligaste konfigurationsfilerna samt viss filstruktur i det upplägg vi senare kommer att beröra i \ref{sec:bestpractices} Best practices.
\section{/etc/ansible}
Under \texttt{/etc/ansible} finner vi systemkonfiguration för styrsystemet samt globalt tillgängliga inventories, roles etc. Vi kommer dock inte att använda mer än \texttt{ansible.cfg} och \texttt{hosts} på en global nivå.
\subsection{ansible.cfg}
Generellt behöver inte den här filen modifieras, men det finns vissa saker som kan vara av intresse även på
en nybörjarnivå.

\texttt{remote\_user} ger oss möjligheten att ansluta med en specifik användare. Vi kommer att beröra detta mer
under best practices.

\texttt{log\_path} kan vara bra att sätta till en lämplig sökväg (ex. \texttt{/var/log/ansible.log}) för att aktivera loggning..

\texttt{vault\_password\_file} ger möjligheten att ange en sökväg för en så kallad 'vault file' - en krypterad fil 
där man kan spara variabler man inte vill ha liggande för allmän åskådan. En sådan fil kan även åkallas på commandline med \texttt{--vault-password-file}.

\texttt{ansible\_managed} är en variabel med samma namn. Denna brukar användas i konfigurationsfiler för att
markera att dessa är konfigurationsstyrda och således inte bör ändras direkt på systemet då dessa ändringar förr 
eller senare kommer att skrivas över.

\texttt{nocows} styr kor. Rekommenderas att sätta till '1' om man inte hyser bovina läggningar.

\subsection{hosts}
\texttt{/etc/ansible/hosts} är den globala host-listan (inventory). Man kan (och eventuellt bör, beroende på miljön i helhet) använda separata hosts-filer och åkalla dessa på kommandoraden.

Inventories är vanligen skrivna i ett init-liknande format, men kan i senare versioner av Ansible även
vara i YAML-format. Vi kommer här att använda det klassiska. Vi kommer att ta vårat exempel från TDDI41 för att
visa hur en sådan kan se ut.

Det vi ser är en enkel enumrering av noderna där vi ger dem funktionella grupperingar (första blocket) där vi
kopplar ihop IP-adresser till dessa grupperingar. Vi skulle lika gärna kunna använda domännamn bör dock sägas, och
skulle vi ha ännu fler noder skulle vi kunna fortsätta lista dessa under grupperingarna. För att demonstrera detta
listar vi även ett subset inaktiva (utkommenterade) clients som inte är aktuella än.

Värt att notera är även att vi kan använda ranges (-). I det här fallet har vi gett ett spann IP-adresser, men skulle vi ha använt oss utav domännamn skulle vi även ha kunnat använda oss av ranges i dessa, exempelvis \texttt{client-[1-2]}. Även \texttt{:} kan användas som \texttt{OR}.

\lstset{language=sh}
\begin{lstlisting}[frame=single]
### BOF ###
## /etc/ansible/hosts
#
## Enumerating nodes

[gateway]
192.168.0.1

[server]
192.168.0.2

[clients]
192.168.0.[3-4]
#192.168.1.[1-50]

## Groupings

[internal:children]
gateway
server
clients

[authnodes:children]
server
clients
### EOF ###
\end{lstlisting}

