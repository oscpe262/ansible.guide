En playbook skrivs i YAML-format. Låt inte detta avskräcka om du aldrig skrivit en rad YAML tidigare, för det är
rätt så lätt att lära sig. Det som är viktigt att komma ihåg är att indentering spelar roll. Det spelar ingen roll 
om du använder mellanslag eller tabbar, men "nivåerna" spelar roll.

En playbook innehåller ``plays'' (den som är bekant med amerikansk fotboll känner igen termerna). 
En play är en mappning av ``tasks'' som skall utföras på en grupp hostar i syfte att nå ett bestämt och känt state.

Vi börjar med ett litet exempel:

\begin{verbatim}
---
- hosts: all
  tasks:
    - name: "Ensure NTP package is installed"
      package:
        name: ntp
        state: present
      notify:
        - restart ntp

  handlers:
    - name: restart ntp
      service:
        name: ntpd
        state: restarted
...
\end{verbatim}

Vi börjar här med en filstart \texttt{---} följt av att vi bestämmer att påföljande tasks skall köras på \texttt{all}, vilket är ett nyckelord och inkluderar samtliga noder i inventoryt. Även \texttt{*} kan användas, och kan användas som wildcard i exempelvis IP-adresser. Vi kan även ange en specifik host eller gruppering. Skulle vi vilja exkludera går även det att göra, precis som vi kan göra en host required, exempelvis:

\verb+webservers:!{{excluded}}:&{{required}}+

Rad tre påbörjar en lista med tasks, och på rad fyra börjar den första (och enda) genom att vi ger den ett \texttt{name}.
Detta är inget som behövs, som tidigare nämnts, men det bör ges. Därefter säger vi att vi skall använda modulen 
\texttt{package} för att säkerställa att paketet \texttt{ntp} är installerat (\texttt{present}). Om denna task körs
och åstadkommer någon förändring på systemet (i det här fallet, installeras) så säger vi till Ansible att anropa en
handler (\texttt{notify}).

Sedan har vi själv handlern, vilket i princip är en task, men den körs endast en gång per play, och endast om den 
anropats till följd av en notify. Hade vi haft en task till som exempelvis justerat en konfigurationsfil med samma notify så hade denna handler körs en (1) gång, även om vi både installerat paketet och justerat konfigurationsfilen, men inte körts om vi direkt efter kört samma playbook en gång till eftersom paketet då varit installerat och konfigurationsfilen varit oförändrad.

Handlern \texttt{restart ntp} i det här fallet använder modulen \texttt{service} för att starta om tjänsten \texttt{ntpd}. Kort om \texttt{service}-modulen kan nämnas att den stödjer de flesta system för daemons som används idag.

Om du inte riktigt greppat konceptet med moduler så är det inget att oroa sig för. Vi kommer att gå in djupare på det vad det lider.

\section{Användare}
Att hålla koll på vilken användare som används på styrenheten likväl som hostarna är ett måste. Vi kan inte anta att
alla som ska köra en playbook har root-access eller ens bör ha. Samtidigt kommer många saker att kräva förhöjda 
rättigheter för att utföras. Likaså kanske inte det konto som används på styrenheten ens finns på (alla) hostarna.

Vi nämnde \texttt{remote\_user} i config-filen tidigare, vilket möjliggör att vi exempelvis kan sätta \texttt{root}
som ``default'', men det är inte alltid att föredra det heller. Best practice vore i stället att ha ett lokalt 
servicekonto på varje nod -- låt oss kalla det för \texttt{ansacc} (kort för ansible account) -- med ssh-nyckel.
Detta servicekonto ges sedan sudorättigheter att bli \texttt{root} (företrädesvis med lösenord vilket sparas i en
vault-fil, men det är överkurs). Därefter kan vi använda oss utav \texttt{become: yes} för att eskalera rättigheterna där det behövs.

\begin{verbatim}
---
- hosts: all
  become: yes
  tasks:
    [...]
...
\end{verbatim}

\section{Taggar}
Inte sällan tar det tid att köra långa plays, inte minst när vi börjar jobba med roles, och då är det onödigt att
behöva mata genom hela batteriet om vi bara vill ändra en enskild sak. Då blir det lätt att man undviker 
konfigurationsstyrning och sedan faller hela poängen. Därför är taggar bra att använda i tasks för att på så sätt
kunna begränsa de tasks som ska köras i en playbook. Så för att kunna urskilja NTP-relaterade prylar genom vårat 
grundläggande körexempel tidigare ger vi vår playbook taggen NTP:

\begin{verbatim}
---
- hosts: all
  tasks:
    - name: "Ensure NTP package is installed"
      package:
        name: ntp
        state: present
      tags: NTP
      notify:
        - restart ntp

  handlers:
    - name: restart ntp
      service:
        name: ntpd
        state: restarted
      tags: NTP
...
\end{verbatim}

Varje task kan ges flera taggar i en kommaseparerad lista (OR:ad, så en räcker för att aktivera).

\section{Importera task lists och variabelfiler}
Ansible tillåter playbooks och task lists att importera andra filer med task lists som kommer att köras in-line. Detta kan
vara smidigt inte minst i roles som ska stödja olika distributioner som kräver olika steg från varandra. Likaså
kan separata filer med variabler.

Dessa två actions är:
\texttt{include\_tasks} och \texttt{include\_vars}. Även \texttt{import\_tasks} och \texttt{import\_vars} kan
användas, med skillnaden att \texttt{import*} preprocessas när playbooken parsas, medan \texttt{include*} processas
när de stöts på under körningen.

\section{Variabler}
Variabler är det som gör Ansible till mer än bara en exekverare av statiska skript.
Variabelnamn startar med en bokstav och kan innehålla [A-Za-z0-9\_]. Märk väl att bindestreck inte är tillåtna.
Variabler är inte helt triviala, och jag rekommenderar starkt att läsa \href{https://docs.ansible.com/ansible/2.6/user_guide/playbooks_variables.html}{\em Ansibles dokumentation} i ämnet. Fortsatt kommer det här dokumentet att förutsätta att läsaren själv slår upp denna dokumentation när tveksamhet råder.
