Roles -- roller -- är egentligen inte mycket annat än playbooks som inte är hostbundna; m.a.o. en template.

Som den uppmärksamme såg tidigare bygger en role på en katalogstruktur där den enda obligatoriska katalogen och 
filen är \texttt{./roles/rolename/tasks/main.yml}. Den filen är i princip en reducerad playbook där vi brutit ut
tasks till en separat fil. Har vi handlers bryter vi ut dem till \texttt{handlers/main.yml}.

Att följa given katalogstruktur gör även att Ansible själv hittar exempelvis handlers, variabler etc.

\section{Variabler i roles}
Variabler i roller kan användas i flera lager med olika prioritet. Lägst prioritet har variabler i 
\texttt{defaults/main.yml}. Dessa kan i sin tur skrivas över av importerade variabler. Fördelen med att ha 
defaults är givetvis att vi inte kommer få problem vid körning på grund av att en variabel inte är deklarerad.
Således bör alla egna variabler vara deklarerade i \texttt{defaults/main.yml}.

Exempelvis skulle våran tidigare NTP-playbook kunna vara en roll som ser ut som följande i role-form med paketnamnet som variabel (vilket är overkill för TDDI41, men vilket vore smidigt i en distromixad miljö):

\begin{verbatim}
### ./roles/ntp/tasks/main.yml ###
---
- name: "Ensure NTP package is installed"
  package:
    name: '{{ ntp_package }}'
    state: present
  tags: NTP
  notify:
    - restart ntp
...

### ./roles/ntp/handlers/main.yml ###
---
- name: restart ntp
  service:
    name: ntpd
    state: restarted
  tags: NTP
...

### ./roles/ntp/defaults/main.yml ###
---
ntp_package: ntp
ntp_server: false
...
\end{verbatim}

\section{Filöverföring}
Att föra över filer från styrhost till nod är inte det vanligaste, men det kan behöva göras emellanåt. Filer
lagrade under \texttt{files} är lättrefererade eftersom Ansible kommer att söka där först och främst.
Främst används dessa av \texttt{copy}-modulen vilken vi inte kommer att beröra här, men återigen är Ansibles
dokumentation lättfunnen.

Vad som är mer intressant är \texttt{templates} där vi kan lagra konfigurationsfilsmallar (vilket vi kommer att 
beröra). Detta är mallar som kan vara scriptade för att dynamiskt kunna anpassa sig utifrån variabler (antingen
insamlade från noden, eller inskickade vid körning) och på så sätt på ett smidigt sätt konfigurera en host.

\section{Roles i playbooks}
För att använda roles måste vi koppla dessa till noder i en playbook. Den uppmärksamme lade även märke till
variabeln \texttt{ntp\_server} ovan. Vi kommer senare att visa hur denna kan användas i en template för att
låta en fil konfigurera både en ntp-server och ntp-klienter. Vi förutsätter dock här att det redan är med i 
tasks-listan och att templaten är på plats, när vi nu visar på hur en playbook som nyttjar detta kan se ut:

\begin{verbatim}
### ./playbook.yml
---
- hosts: gateway
  become: yes
  roles:
    - { role: ntp, tags: ["NTP","gateway"], ntp_server: true }

# Här ser vi en alternativ formattering som är motsvarande den ovan. 
# Vilken man vill använda är en smaksak.

- hosts: authnodes
  become: yes
  roles:
    - role: ntp
      tags: NTP,authnodes
      ntp_server: false
\end{verbatim}

Den uppmärksamme har säkerligen även ifrågasatt varför vi har med \texttt{ntp\_server: false} när den är deklarerad
som \texttt{false} i \texttt{defaults/main.yml}. Det är en god iakttagelse, och den enda anledningen är för att 
visa på de olika formatteringarna. Det skadar givetvis inte att skicka med den, och det skulle kunna vara att 
rekommendera utifall att vi av någon anledning skulle få för oss att ändra default-värdet vid senare tillfälle.
